\documentclass[9pt,a4paper,twoside]{tau}
\usepackage[english]{babel}
\usepackage{tauenvs}

%----------------------------------------------------------
% TITLE
%----------------------------------------------------------

\title{Investigative Report of Franklin Resources Inc. [stock:BEN]from April 2022 to April 2024}

%----------------------------------------------------------
% AUTHORS, AFFILIATIONS AND PROFESSOR
%----------------------------------------------------------

\author[a,1]{Bucsa, Justin}

%----------------------------------------------------------

\affil[a]{Stealth}
\professor{}

%----------------------------------------------------------
% FOOTER INFORMATION
%----------------------------------------------------------

\institution{Stealth}
\ftitle{}
\date{April 19, 2024}
\etal{Bucsa}
\course{}

%----------------------------------------------------------
% ABSTRACT
%----------------------------------------------------------

\begin{abstract}    
    This report provides a comprehensive analysis of Franklin Resources Inc. (BEN) as a potential investment opportunity. It examines the company's performance over a two-year period (April 2022 - April 2024), encompassing historical background, strategic goals, and major shareholder structure. We will delve into market data to understand BEN's price movements and analyze key events that transpired within the past year (April 2023 - April 2024) to assess their impact. Finally, the report culminates in a holistic evaluation of Franklin Resources Inc.'s investment potential.
\end{abstract}

%----------------------------------------------------------
% \keywords{}
%----------------------------------------------------------

\begin{document}
		
	\maketitle
	\thispagestyle{firststyle}
	\tauabstract
	\tableofcontents

%----------------------------------------------------------

\section{Introduction}

    \taustart{F}ranklin Resources Inc. [NYSE:BEN], is one of the world's largest investment managers, is better known as Franklin Templeton. It currectly over sees \$1.6 trillion in total assets under management, over 1400 investment professionals in 25 counties and over 9000 employees globally. In this report will first discuss

\section{History}
    Franklin Resources boasts a rich history dating back to 1947, when it began operations in the investment management field. The company initially focused on mutual funds, offering both fixed income and equity options (growth and value-oriented). Through strategic acquisitions, Franklin has continuously expanded its reach and expertise. Notable additions include:

    - 1992: Templeton, a global investment firm, bringing international investment capabilities.
    
    - 1996: Franklin Mutual Series, further solidifying its mutual fund portfolio.

    - 2000 \& 2001: Acquisitions like Franklin Bissett Canadian and Fiduciary Trust International broadened their offerings to include Canadian investment management and trust services.
    
    - 2019-2022: A period of significant expansion, with acquisitions like Benefit Street Partners (alternative credit), Athena Capital Advisors (wealth management), Legg Mason (global investments), O’Shaughnessy Asset Management (quantitative assets), Lexington Partners (alternative investments), and Alcentra (alternative credit).
    
    This strategic growth through acquisitions has transformed Franklin Resources into a leading global investment management firm with a diverse range of offerings to meet evolving investor needs.
    
\section{Events}

    \subsection{May 2023 - Acquisition of Putnam Investments Announced}
	
        Franklin Resources Inc. announced a significant acquisition in May 2023, entering into a definitive agreement to acquire Putnam Investments from Great-West Lifeco. Inc. (Great-West), a member of the prominent Power Corporation group. This strategic move strengthens Franklin Templeton's position in the asset management industry. Great-West and Power Corporation are leaders in global insurance, retirement, wealth management, and asset management, aligning well with Franklin Templeton's existing strengths. The acquisition was expected to close in the first quarter of fiscal year 2024, subject to customary closing conditions.

    \subsection{January 2024 - Acquisition of Putnam Investments Completed}
	
        Franklin Resources Inc. successfully completed the acquisition of Putnam Investments, expanding its global footprint and AUM (Assets Under Management). This strategic move solidified Franklin Resources' position as a leading investment management firm.

    \subsection{Table of contents}

        The \textit{tau class} provides a table of contents. Each level of the ToC provides a preview of the content and its location in the document.

    \subsection{Tau start}

        We included the \verb*|\taustart{}| command, which provides a personalized lettrine for the beginning of a paragraph.
        
    \subsection{Caption}

        \subsubsection{Figures}

            The provided \verb*|\captionsetup[figure]| command customizes the appearance of captions for figures in \LaTeX\ documents. For example, in Fig. \ref{fig:figure}, shows an example figure.
			
            \begin{figure}[H]
                \centering
                \includegraphics[width=0.8\columnwidth]{Figures/Example.pdf}
                \caption{Example figure (obtained from \textit{PGFPlots - A LaTeX package to create plots}. [Online]. Available: \url{https://pgfplots.sourceforge.net/}).}
                \label{fig:figure}
            \end{figure}

        \subsubsection{Tables}
    
            The \verb*|\captionsetup[table]| command customizes the appearance of the captions for tables in the document. The \verb*|\tabletext{}| is used to add notes to tables easily. Table \ref{tab:table}, shows an example table.
            
            \begin{table}[H]
                \centering
                \caption{Small table example.}
    		\label{tab:table}
                \begin{tabular}{cc}
            	\toprule
                    \textbf{Column 1} & \textbf{Column 2} \\
                    \midrule
                    Data 1 & Data 2 \\
                    Data 3 & Data 4 \\
                    \bottomrule
                \end{tabular}
                    
                \tabletext{Note: I'm a table text for additional information.}
                    
            \end{table}

    \subsection{Equation}
    
        Equation \ref{ec:equation} shows an example equation. 

	\begin{equation} \label{ec:equation}
            \frac{\hbar^2}{2m}\nabla^2\Psi + V(\mathbf{r})\Psi = -i\hbar \frac{\partial\Psi}{\partial t}
	\end{equation} 

        The \textbf{amssymb} package was not necessary to include, because the stix2 font incorporates mathematical symbols for writing quality equations. In case you choose another font, uncomment the package in \textit{tau class} code.

        If you want to change the values that adjust the spacing above and below in the equations, go to \textit{tau class-math packages} section and play with \verb|\setlength{\eqskip}{6.5pt}| value until the preferred spacing is set. See appendix for more information.

\section{Environment}

    The \textit{tau class} includes custom environments designed to enhance the presentation of information within documents. Among these custom environments are \textbf{tauenv}, \textbf{info} and \textbf{note}.

    \begin{tauenv}[frametitle=Custom title]
        This is an example of the custom title environment. To add a title type \verb|[frametitle=Custom title]| next to the beginning of the environment (as shown in this example).
    \end{tauenv}

    One of the main features of the info and note environment is that they automatically change the language of their titles (currently English and Spanish).

\section{Coding}

    \textit{Tau class} includes the \textit{listings} package, which offers versatile and customizable features for typesetting code snippets in \LaTeX\ documents. Specifically for C, C++, \LaTeX\ and Matlab codes. 

    For C and C++ codes, the \textit{listings} package recognizes the syntax of these programming languages and highlights keywords, comments, and string literals accordingly.

    \lstinputlisting[caption=Example of C code., language=C]{example.c}

    Similarly, for Matlab codes, the \textit{listings} package offers syntax highlighting and line numbering, to the MATLAB language syntax.
    
    \lstinputlisting[caption=Example of matlab code., language=Matlab]{example.m}

\section{References}

    The default formatting for references follows the IEEE style. This style is commonly used for technical documents, research papers, and scholarly articles in engineering fields \cite{einstein}.

    At the end of the document, you will find an example of the default reference formatting \cite{dirac}.
        
\section{Appendix}

    \subsection{Environments preview}

        The following environments are defined in \textit{tauenvs} package.
		
		\subsubsection{Tau environment}

                The following code defines the tauenv environment. A custom title can be added to this environment.

			\begin{tauenv}[frametitle=Tauenv]
                    Lorem ipsum dolor sit amet, consectetur adipiscing elit. Sed vestibulum justo quis massa aliquet, ut ultrices quam bibendum.
			\end{tauenv}
		
\begin{lstlisting}[language=TeX, caption=Tauenv environment code.]
\newmdenv[
	backgroundcolor=taublue!22, 					
	linecolor=taublue,									
	linewidth=0.7pt,
	frametitle=\vskip0pt\bfseries,
	frametitlerule=false,
	frametitlefont=\color{taublue}\bfseries\sffamily,
	frametitlealignment=\raggedright,
	innertopmargin=3pt,
	innerbottommargin=6pt,
	innerleftmargin=6pt,
	innerrightmargin=6pt,
	font=\selectfont,
	fontcolor=taublue,									
	frametitleaboveskip=8pt,
	skipabove=10pt
]{tauenv} \end{lstlisting}
		
		\subsubsection{Note}

                This code defines the note environment.

  			\begin{note}
                    Lorem ipsum dolor sit amet, consectetur adipiscing elit. Sed vestibulum justo quis massa aliquet, ut ultrices quam bibendum.
			\end{note}
		
\begin{lstlisting}[language=TeX, caption=Note environment code.]
\newmdenv[
	backgroundcolor=taublue!22, 						
	linecolor=taublue,									
	linewidth=0.7pt,
	frametitle=\vskip0pt\bfseries\notelanguage,
	frametitlerule=false,
	frametitlefont=\color{taublue}\bfseries\sffamily,
	frametitlealignment=\raggedright,
	innertopmargin=3pt,
	innerbottommargin=6pt,
	innerleftmargin=6pt,
	innerrightmargin=6pt,
	font=\normalfont,
	fontcolor=taublue,									
	frametitleaboveskip=3pt,
	skipabove=10pt
]{note} \end{lstlisting}

		\subsubsection{Info}

                This code defines the info environment.

    		\begin{info}
                    Lorem ipsum dolor sit amet, consectetur adipiscing elit. Sed vestibulum justo quis massa aliquet, ut ultrices quam bibendum.
			\end{info}
		
\begin{lstlisting}[language=TeX, caption=Info environment code.]
\newmdenv[
	backgroundcolor=taublue!22, 						
	linecolor=taublue,									
	linewidth=0.7pt,
	frametitle=\vskip0pt\bfseries\infolanguage,
	frametitlerule=false,
	frametitlefont=\color{taublue}\bfseries\sffamily,
	frametitlealignment=\raggedright,
	innertopmargin=3pt,
	innerbottommargin=6pt,
	innerleftmargin=6pt,
	innerrightmargin=6pt,
	font=\normalfont,
	fontcolor=taublue,									
	frametitleaboveskip=3pt,
	skipabove=10pt
]{info} \end{lstlisting}

    \subsection{Alternative title}

         You can make the following modification to \textit{tau class} in the \textit{title preferences} section to change the position of the title. This will move the title to the left. 

\begin{lstlisting}[language=TeX, caption=Alternative title.]
\renewcommand{\@maketitle}{%
        \vskip-18pt
    {\RaggedRight\bfseries\color{taublue}\fontsize{18}{22}\sffamily\selectfont\@title\par}
		\vskip8pt
    {\RaggedRight\normalsize\sffamily\@author\par}
        \vskip8pt
    {\RaggedRight\fontsize{7pt}{8pt}\selectfont\@professor\par}
        \vskip24pt
}% 
\end{lstlisting}

    \subsection{Equation skip value}

        Play with the value of \verb|\eqskip| until the preferred spacing is set for equations.

\begin{lstlisting}[language=TeX, caption=Equation skip code.]
\newlength{\eqskip}\setlength{\eqskip}{6.5pt}
\expandafter\def\expandafter\normalsize\expandafter{%
    \normalsize%
    \setlength\abovedisplayskip{\eqskip}%
    \setlength\belowdisplayskip{\eqskip}%
    \setlength\abovedisplayshortskip{\eqskip-\baselineskip}%
    \setlength\belowdisplayshortskip{\eqskip}%
}
\end{lstlisting}
					
%----------------------------------------------------------

\addcontentsline{toc}{section}{References}
\printbibliography

%----------------------------------------------------------

\begin{center}
	\vskip10pt
	Enjoy writing with tau \LaTeX\ class $\blacksmiley$ \\ 
	\vskip10pt
	\textit{Contact:} \\
	\faLink\ \href{https://sites.google.com/view/memo-notess/p%C3%A1gina-principal}{https://sites.google.com/memo-notess} \\
	\faEnvelope[regular]\ memo.notess1@gmail.com \\
	\faInstagram\ memo.notess\\
\end{center}

%----------------------------------------------------------

\end{document}